\documentclass[11pt]{article}
\usepackage[a4paper,margin=1.8cm]{geometry}
\usepackage{amsmath, amssymb, amsthm}  % Essential math packages
\usepackage{graphicx}                   % For figures
\usepackage{hyperref}                   % Clickable links
\usepackage{parskip} % cleaner spacing
\usepackage{enumitem}
\usepackage{amssymb}
\documentclass{article}
\usepackage{blindtext}
\usepackage{titlesec}
\title{Sections and Chapters}

\newtheorem{definition}{Definisjon}

\title{Oblig 1}
\author{Oliver Ekeberg}
\date{\today}

\begin{document}
\maketitle


\tableofcontents


\section{1.4}

\subsection{1.4.2}

Vis at standardbasisen $(e_1, ..., e_n)$ er en basis for $\mathbb{K}^n$

pf:

En 

\vspace{1em}
Først viser jeg at $(e_1,...,e_n)$ er lineært uavhengig. Dette skjer når likningen

\begin{align*}
    0 = \alpha_1 e_1 + ... + \alpha_n e_n     
\end{align*}

har eneste løsning at $\alpha_1=...=\alpha_n=0$. Vi vet videre at $e_j &= (\delta_{jk})^{n}_{k=1}$ hvor j er da posisjonen til 
vektoren inne i listen, og k er da den k´te posisjonen i n-tuppelen. 

I nesten enhver situasjon så har vi en liste med n-tupler

\begin{align*}
    0 &= \alpha_1 e_1 + ... + \alpha_n e_n \\
    &= \sum_{j=1}^{n} \alpha_j * e_j \\
    &= \sum_{j=1}^{n} \alpha_j * (\delta_{jk})^{n}_{k=1} \\
    &= (\sum_{j=1}^{n} \alpha_j \delta_{jk})^n_{k=1} \\
    &= (\alpha_k)
\end{align*}
hvis og bare hvis $\alpha_1= ... = \alpha_n = 0$  

Så $(e_1,...e_n)$ er lineært uavhengig

\vspace{1em}
Deretter må jeg vise at listen $(e_1,...e_n)$ utspenner heel $\mathbb{K}^n$

Definisjonen på spennet av en liste med vektorer er definert som

$$
span(\mathbb{K}^n) = \{ \alpha_1 e_1 + ... + \alpha_n e_n  : \alpha_1,...,\alpha_n \in \mathbb{K} \}
$$

Siden $(e_1,...,e_n)$ er lineært uavhengig, så vet vi at det finnes n uavhengige vektorer, og de må av den grunn utspenne hele $\mathbb{K}^n$




\subsection{1.4.6}
\subsubsection{a)}

La $C = (e_1,..., e_n)$ være standardbasisen til $\mathbb{K}^n$. Vis at $[ x ]_C = x, \forall x \in \mathbb{K}^n$.

pf:

Vi vet at $e_j = (\delta_{jk})$ som er den j-te n-tuppelen, som er 0 overalt enn i den k-te posisjonen.

Da er enhver $x \in \mathbb{K}^n$ en lineær kombinasjon av standardbasisen, og kan skrives som

\begin{align*}
    x = \sum_{j=1}^{n} x_j e_j \\
    &= \sum_{j=1}^{n} x_j (\delta_{jk}) \\
    &= (x_k)
\end{align*}

\vspace{1em}

Da vil av teorem $1.4.5$, som sier at hvis $u = x_1 u_1+...+x_n u_n => [u]_B = (x_1,...,x_n)$ gitt at basisen til et vektorrom $U$ over $\mathbb{K}$ er gitt ved at $B=(u_1,...,u_n)$, kunne si at
$[x]_C = x,  \forall x \in \mathbb{K}^n$



\subsubsection{b)}

La $U$ være et vektorrom over $\mathbb{K}$ med en basis $B=(u_1,...,u_n)$. Vis at 

$$
[u_j]_B = e_j, \forall j =1,...,n
$$

Vi vet at vi kan uttrykke $u_j$ som $0 * u_1 + ... + 1 * u_j + ... + 0 * u_n$.

Da vet vi fra teorem $1.4.5$  at $[u_j]_B = (0,...,1,...,0)$. Dette er det samme som å skrive $e_j = (0,...,1,...,0)$ hvor $1$ er i den j-te plassen, som viser at

$[u_j]_B = e_j, \forall j = 1,...,n$





\end{document}
