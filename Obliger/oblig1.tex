\documentclass[11pt]{article}
\usepackage[a4paper,margin=1.8cm]{geometry}
\usepackage{amsmath, amssymb, amsthm}  % Essential math packages
\usepackage{graphicx}                   % For figures
\usepackage{hyperref}                   % Clickable links
\usepackage{parskip} % cleaner spacing
\usepackage{enumitem}
\usepackage{amssymb}

\title{Sections and Chapters}

\newtheorem{definition}{Definisjon}

\title{Oblig 1}
\author{Oliver Ekeberg}
\date{\today}

\begin{document}
\maketitle


\tableofcontents


\section{1.4}

\subsection{Oppgave 1.4.2}

Vis at standardbasisen $(e_1, ..., e_n)$ er en basis for $\mathbb{K}^n$

pf:

En 

\vspace{1em}
Først viser jeg at $(e_1,...,e_n)$ er lineært uavhengig. Dette skjer når likningen

\begin{align*}
    0 = \alpha_1 e_1 + ... + \alpha_n e_n     
\end{align*}

har eneste løsning at $\alpha_1=...=\alpha_n=0$. Vi vet videre at $$e_j = (\delta_{jk})^{n}_{k=1}$$ hvor j er da posisjonen til 
vektoren inne i listen, og k er da den k´te posisjonen i n-tuppelen. 

I nesten enhver situasjon så har vi en liste med n-tupler

\begin{align*}
    0 &= \alpha_1 e_1 + ... + \alpha_n e_n \\
    &= \sum_{j=1}^{n} \alpha_j * e_j \\
    &= \sum_{j=1}^{n} \alpha_j * (\delta_{jk})^{n}_{k=1} \\
    &= (\sum_{j=1}^{n} \alpha_j \delta_{jk})^n_{k=1} \\
    &= (\alpha_k)
\end{align*}
hvis og bare hvis $\alpha_1= ... = \alpha_n = 0$  

Så $(e_1,...e_n)$ er lineært uavhengig

\vspace{1em}
Deretter må jeg vise at listen $(e_1,...e_n)$ utspenner heel $\mathbb{K}^n$

Definisjonen på spennet av en liste med vektorer er definert som

$$
span(\mathbb{K}^n) = \{ \alpha_1 e_1 + ... + \alpha_n e_n  : \alpha_1,...,\alpha_n \in \mathbb{K} \}
$$

Siden $(e_1,...,e_n)$ er lineært uavhengig, så vet vi at det finnes n uavhengige vektorer, og de må av den grunn utspenne hele $\mathbb{K}^n$




\subsection{Oppgave 1.4.6}
\subsubsection{1.4.6 a)}

La $C = (e_1,..., e_n)$ være standardbasisen til $\mathbb{K}^n$. Vis at $[ x ]_C = x, \forall x \in \mathbb{K}^n$.

pf:

Vi vet at $e_j = (\delta_{jk})$ som er den j-te n-tuppelen, som er 0 overalt enn i den k-te posisjonen.

Da er enhver $x \in \mathbb{K}^n$ en lineær kombinasjon av standardbasisen, og kan skrives som

\begin{align*}
    x = \sum_{j=1}^{n} x_j e_j \\
    &= \sum_{j=1}^{n} x_j (\delta_{jk}) \\
    &= (x_k)
\end{align*}

\vspace{1em}

Da vil av teorem $1.4.5$, som sier at hvis $u = x_1 u_1+...+x_n u_n => [u]_B = (x_1,...,x_n)$ gitt at basisen til et vektorrom $U$ over $\mathbb{K}$ er gitt ved at $B=(u_1,...,u_n)$, kunne si at
$[x]_C = x,  \forall x \in \mathbb{K}^n$



\subsubsection{1.4.6 b)}

La $U$ være et vektorrom over $\mathbb{K}$ med en basis $B=(u_1,...,u_n)$. Vis at 

$$
[u_j]_B = e_j, \forall j =1,...,n
$$

Vi vet at vi kan uttrykke $u_j$ som $0 * u_1 + ... + 1 * u_j + ... + 0 * u_n$.

Da vet vi fra teorem $1.4.5$  at $[u_j]_B = (0,...,1,...,0)$. Dette er det samme som å skrive $e_j = (0,...,1,...,0)$ hvor $1$ er i den j-te plassen, som viser at

$[u_j]_B = e_j, \forall j = 1,...,n$





\section{1.5}

\subsection{Oppgave 1.5.2}


\subsubsection{1.5.2 a)}
 
Vis at dim \( P_n = n + 1 \)

pf: Siden dimensjonen til en basis er lik antall vektorer i den basisen, må jeg vise at basisen $ P_n $  = \{ alle reelle polynomer av grad høyst n \} har \( n+1 \) vektorer.

Videre er ethvert polynom av grad n formulert som

\begin{align*}
    p(t) &= \alpha_{0}^{} + \alpha_{1}^{} t + ... + \alpha_{n}^{} t^n \\
    &= \alpha_{0}^{} p_0(t) + \alpha_{1}^{} p_1(t) + ... + \alpha_{n}^{} p_n(t) 
\end{align*}


Så lineær kombinasjonen av enhver p(t) $\in P_n$ inneholder da $ n+1 $ vektorer, og $ span(p_0, ..., p_n) = P_n $ . Logikken er det samme som at settet $C_4 = {0, 1, 2, 3, 4} $ inneholder  5 tall, men har input lik 4

Må nå bare vise at $ p(t) = 0 $ hvis $ \alpha_{0}^{} = .. = \alpha_{n}^{} = 0 $. Hvis det ikke er slik, betyr det at noen av $ p_0, ..., p_n $ er lineært uavhengige, og da er dimensjonen mindre. enn n+1.

\[
     0 = \sum_{ j=0 }^{ n } \alpha_{j}^{} p_j
\] 
Hvis alle $ p_j = 0 $, så er ikke basisen som $ P_n $ består av lin. uavh. Derfor må alle $ \alpha_{j}^{} = 0 for j=1,..., n  $



\subsubsection{1.5.2 b)}


Vis at dim $ P = \infty $ 

pf:

P := (alle reelle polynomer)

Anta at P har en endeligdimensjonal, med basis $ B := (p_0(t), ..., p_n(t)) $ . Det vil si at for en $ p(t) \in P $, så er $ p(t) = t^n = \alpha_{0}^{} p_0(t) + \alpha_{1}^{} p_1(t) + ... + \alpha_{n}^{} p_n(t)$. Men siden $ p(t) \in P $, så vil også $ p(t) = t^{n+1} \in P $, og $ t^{n+1} = \alpha_{0}^{} p_0 + .. + \alpha_{n}^{} p_n $ .  .Dette motsier argumentet om at P skal være endelig dim. fordi basisen til P skal utspenne hele P. Altså $ span(B) = P $. Av den grunn, er P uendelig dim., og dim P = $ \infty $.


\subsubsection{1.5.2 c)}

$ C^0 (\mathbb{R}, \mathbb{R})$ er rommet som inneholder alle kontinuerlige funksjoner s.a. $ f: \mathbb{R} \rightarrow \mathbb{R} $  

pf:

P er et underrom av $ C^0 (\mathbb{R}, \mathbb{R})$ fordi ethvert polynom er en kontinuerlig funksjon. Men P inneholder ikke alle kontinuerlige funksjoner, for eksempel er ikke $ f(x) = sin(x) $ et polynom. Derfor er P et ekte underrom av $ C^0 (\mathbb{R}, \mathbb{R})$. Og siden dim P = $ \infty $, så må dim $ C^0 (\mathbb{R}, \mathbb{R})$ også være uendelig.



\section{1.6}

\subsection{Oppgave 1.6.5}

\subsubsection{1.6.5 a)}

$ V \oplus W := \{ (v,w) : v \in V, w \in W \} $ 

pf:

Vi sier at $ V \oplus W $ er et vektorrom fordi 


\begin{itemize}
    \item vi utstyrer det med nullelementet $ 0_{V \oplus W} = (0_V, 0_W) $
    \item for to elementer $ (v_1, w_1), (v_2, w_2) \in V \oplus W $ , så definerer vi addisjon som $ (v_1, w_1) + (v_2, w_2) = (v_1 + v_2, w_1 + w_2) $
    \item skalar multiplikasjon som $ \alpha_{}^{} (v, w) = (\alpha_{}^{}  v, \alpha_{}^{}  w) $ for en skalar $ \alpha_{}^{}  $ 
\end{itemize}

Fra disse operasjonene kan vi se at alle åtte aksiomene for et vektorrom er oppfylt, fordi både V og W er vektorrom. Derfor er $ V \oplus W $ et vektorrom.


Eksempel: Punkt 5:


For en $ u \in V \oplus W $, og en $ \alpha_{}^{} , \beta \in \mathbb{K} $ (V og W er vektorrom over $ \mathbb{K} $ ), så er 

\begin{align*}
    \alpha_{}^{} (\beta u) = \alpha_{}^{} (\beta (v, w)) = \alpha_{}^{} (\beta v, \beta w) \\
    &= (\alpha_{}^{} (\beta v), \alpha_{}^{} (\beta w)) \\
    &= ((\alpha_{}^{} \beta) v, (\alpha_{}^{} \beta) w) \\
    &= (\alpha_{}^{} \beta)(v, w)
\end{align*}

\subsubsection{1.6.5 b)}


Siden $ (v_1, 0_W), ... , (v_n, 0_W) \in V $ og $ (0_V, w_1), ... , (0_V, w_m) \in W$, så er enhver $ (v, w) \in V \oplus W $ en lineær kombinasjon av disse vektorene. Derfor spenner de opp hele $ V \oplus W $. Og $ ((v_1, 0_W), ..., (v_n, 0_W), (0_V, w_1, ..., (0_V, w_m))) $ blir da en basis for $ V \oplus W $.    


\subsubsection{1.6.5 c)}

Dimensjonen til et vektorrom er lik antall vektorer i en basis for det rommet. Fra oppgave b) så vet vi at en basis for $ V \oplus W $ er gitt ved $ ((v_1, 0_W), ..., (v_n, 0_W), (0_V, w_1, ..., (0_V, w_m))) $. Denne listen inneholder n + m vektorer. Derfor er dim $ V \oplus W = n + m = dim V + dim W $


\section{1.8}

\subsection{Oppgave 1.8.2}

\subsubsection{1.8.2 a)}

Vis at vektorrommet $ \mathbb{K}^n $ har dim n. 


pf:

Standardbasisen til $ \mathbb{K} ^n$ er $ C = (e_1, ..., e_n) $ . Må vise at denne listen er linært uavhengig.

En liste er linært uavhengig hvis likningen 

\[
     0 = \sum_{ j=1 }^{ n } \alpha_{j}^{} e_j
\] 

Vet fra tidligere av at $ e_j := (\delta_{jk})_{k=1}^n $ 

\[
     0 = \sum_{ j=1 }^{ n } \alpha_{j}^{} (\delta_{jk})_{k=1}^n = (\sum_{ j=1 }^{ n } \alpha_{j}^{} \delta_{jk})_{k=1}^n = (\alpha_k)_{k=1}^n
\] 

Alle $ \alpha_k $ må være lik 0. Videre må jeg vise at listen spenner opp hele $ \mathbb{K}^n $.


Ta en $ x = (x_1, ..., x_n) \in \mathbb{K}$. Da er $ x = x_1e_1 + ... + x_n e_n $. Av dette, er da $ span(C) = \mathbb{K}^n $, og dim $ \mathbb{K}^n = n$   



\subsubsection{1.8.2 b)}

$ M_{mxn}(\mathbb{K})  = (M_{jk})_{j=1,..m, k=1,..n}$. er mengden av alle $ mxn $ matriser med elementer i $ \mathbb{K} $. Siden $ M_{jk} $ er ethvert element i matrisen, så vil en lin. komb. av $ M_{mxn} (\mathbb{K}) $ være 0 iff. $ 0 = (\alpha_{}^{} M)_{jk} $ har eneste løsning at $ \alpha_{jk}^{} =0 \forall j=1,...,m$ og $ k=1,...,n $. Så lenge $ \alpha_{jk}^{} \in \mathbb{K} $


\subsection{Oppgave 1.8.3}



\end{document}
